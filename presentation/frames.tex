


\begin{frame}[noframenumbering]
	\titlepage
\end{frame}


\begin{frame}{Постановка задачи}
	\begin{itemize}
		\item \textbf{Задачей} данной работы является прогнозирование потребности экономики в квалифицированных инженерах-мехатроников в 2025 году. Работа подразумевает 
		описание выбранного подхода для оценки.
		
		\item \textbf{Результатом} данной работы станет расчитанное число мехатроников в 2025 году.
	\end{itemize}
\end{frame}

\begin{frame}{Мехатроник}
	Инженер-мехатроник - собирательное понятие для специальностей:
	\begin{itemize}
		\item Разработчик АСУ ТП.
		\item Робототехник.
		\item Специалисты в автомобильной, авиационной и космической техники и другие.
	\end{itemize}
\end{frame}

\begin{frame}{Методология} 

	\begin{center}
		\LARGE
		$L_{e, t+1} = \frac{X_{e, t+1}}{F_{e, t+1}} * \frac{F_{e, t}}{X_{e, t}} * L_{e, t}$
	\end{center}
	\par
	$L_{e,t}$ – численность занятых в экономике по ВЭДу e в год t;
	\par
	$F_{e,t}$ – производительность труда по ВЭДe в год t; 
	\par
	$X_{e,t}$ – валовая добавленная стоимость  ВЭДа в год.

\end{frame}

\begin{frame}{Методология} 
	
	\par
	\begin{center}
		\LARGE
		$\delta D = \delta L_{e, t+1} + L_{e, t+1}^{-}$
	\end{center}
	\par
	$\delta D$ - ежегодная дополнительная потребность в кадрах.
	\par
	$L_{e, t+1}^{-}$ - численность работников "на замену".
	\par
	$\delta L_{e, t+1}$ - количество кадров "на рост".
\end{frame}

\begin{frame}{Оценка} 
	\begin{itemize}
		\item Данные для расчета $L_{e, 2025}$ берем с сайта Росстата.
		\item Считаем, что инженеры-мехатроники работают в отрасли \textit{"Обрабатывающие производства" подраздела "производство машин и
		оборудования"}
		\item Для оценки $L_{e, t+1}^{-}$ предполагаем, что доля работающих пенсионеров среди мехатроников 20\%.
	\end{itemize}
\end{frame}

\begin{frame}{Результат} 
		\begin{table}
		\begin{center}
			\caption{Численность работающих в ВЭД, тыс. человек}\label{res}
			\begin{tabular}{ |c|c|c|c| } 
				\hline
				Отрасль                                        & 2020   & 2025 \\
				\hline
				Обрабатывающая пром.                           & 9713,5 & 9459,6 \\
				\hline
				в частности, производство машин и оборудования & 792    & 771 \\
				\hline
			\end{tabular}
		\end{center}
	\end{table}
	К 2025 году нужно выпустить \textbf{138 тыс. инженеров}, чтобы покрыть потребности рынка труда.
\end{frame}