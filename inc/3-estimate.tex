\section{Оценка}
Согласно документу "Общероссийский классификатор видов экономической деятельности" несуществует
отельного ВЭД - махатроника. Но из этого же документа узнаем, что эта специальность больше подходит под класс "Обрабатывающие производства". 
Потому возьмем из таблиц $X_{e,2020} = 11`169`334 $ миллионов рублей, $L_{e,2020} = 9713,5$ тыс. человек. Величины $X_{e,2025}$ и $\frac{F_{e, 2020}}{F_{e, 2025}}$ нам неизвестны, потому попытаемся их спрогнозировать.
\par
Сначала найдем $\frac{F_{e, 2020}}{F_{e, 2025}}$. Прирост индекса производительности труда в экономике последние 3 года, включая 2020, составлял около $3.5\%$ в год\cite{work-perfomance}. 
Предположим, что этот тренд сохранится до 2025 года. Вычислим индекс
производительности труда в 2025 году в процентах от 2020:
\par
$\frac{`F_{e, 2025}}{`F_{e, 2020}} = 100\%*(1 + a)^4 = 100\%*(1 + 0.035)^4 \approx 114.75\%$

В формуле выше за 100\% берем значение индекса в 2020 году. Тогда,
\par
$\frac{F_{e, 2020}}{F_{e, 2025}} = (\frac{1}{100} * \frac{`F_{e, 2025}}{`F_{e, 2020}})^{-1} \approx 0.87$
\par
Теперь вычислим $X_{e,2025}$. Для этого воспользуемся простой линейной аппроксимацией. 
Значения за предыдущие года берем таблицы \cite{fedstat}:

\begin{table}
\begin{center}
	\caption{Валовая добавленная стоимость (миллионы рублей) по годам}\label{VDS}
	\begin{tabular}{ |c|c|c|c|c|c|c| } 
		\hline
		Года & 2017       & 2018       & 2019       & 2020       & 2025 \\
		\hline
		ВДС  & 10456681.4 & 10871001.8 & 11172128.4 & 11169334.6 & 12502692.6 \\
		\hline
	\end{tabular}
\end{center}
\end{table}

Теперь можем вычислить численность занятых в экономике по виду экономической деятельности:
\par
$L_{e, 2025} = \frac{X_{e, 2025}}{F_{e, 2025}} * \frac{F_{e, 2020}}{X_{e, 2020}} * L_{e, 2020} = \frac{12`502`692}{11`169`334} * 0.87 * 9'713'500 \approx 9`459`566.9$ мил. чел.
\newline
\textbf{Заметим, что $L_{e, 2025} -  L_{e, 2020} \approx -253933$, то есть число людей, занятых в 
обрабатывающих производствах сократится.} Отметим, что это число согласуется с таблицей\cite{workers-count} в которой видно, что последние года количество работающих в данной области сокращалось.
Исходя из этого, можно сделать вывод, что выпускники в основном будут востребованы как замена уходящих сотрудников на уже существующих рабочих местах, а не на дополнительно созданные.
\par
Теперь необходимо вычислить, какая доля работающих в отрасли "Обрабатывающие производства" являются мехатрониками. Для этого вновь обратимся к данным Росстата. Согласно таблице "Численность принятых работников списочного состава в Российской Федерации по видам экономической деятельности"\cite{work-count-VED} число принятых работников в "Обрабатывающие производства" 
составило 1.3 миллиона человек, а в подраздел "производство машин и оборудования" 106 тысяч человек.
Будем считать, что отношение этих чисел год от года меняется слабо, и оно отображает процент 
мехатроников среди числа остальных. То есть, инженеров-мехатроников сейчас в стране:
\par
$L_{e,2020}* \frac{106}{1300} = 9`713`500 * \frac{106}{1300} \approx 792024 $.
\newline
А в 2025 году будет:
\par
$L_{e,2025}* \frac{106}{1300} = 9`459`567 * \frac{106}{1300} \approx 771318 $.

Наконец мы можем расчитать ежегодную дополнительную потребность в кадрах. 
Счетная палата РФ сообщает, что число работающих пенсионеров в России достигает 20\% от
числа трудоспособных. Будем считать, что и для анализируемой отрасли этот процент справедлив.
В таком случае $k_{e,t} = 0.2$.
\par
$L_{e, 2025}^{-} =  L_{e, 2020} * k_{e, 2020} = 792024 * 0.2 \approx 158405$

$\delta L_{e, 2025} = L_{e, 2025} - L_{e, 2020} = 771318 - 792024 \approx -20706$

$\delta D = \delta L_{e, 2025} + L_{e, 2025}^{-} = -20706 + 158405 \approx 138000$

Итого, число инженеров-мехатроников, которое необходимо выпустить -- 138 тысяч человек.
\clearpage