\section{Методика}
Инженер-мехатроник - довольно обширная профессия. Сюда входят специалисты
отрасли производства компьютеров, электронных и оптических изделий, а так же
инженеры-робототехники, программисты АСУ ТП, инженеры по производству и
ремонту автомобилей. Для простоты расчета будем считать, что это кадры, работающие
в отрасли "Обрабатывающие производства" подраздела "производство машин и
оборудования" согласно Общероссийскому классификатору видов экономической деятельности\cite{economic-action-types}.

Численность занятых в экономике по виду экономической деятельности
можно расчитать по соотношению: 
\par
$L_{e, t+1} = \frac{X_{e, t+1}}{F_{e, t+1}} * \frac{F_{e, t}}{X_{e, t}} * L_{e, t}$
\newline
где $L_{e,t}$ – численность занятых в экономике по виду экономической деятельности e в год t;
$F_{e,t}$ – производительность труда по ВЭДe в год t; $X_{e,t}$ – валовая добавленная стоимость (ВДС) ВЭДа в год. Эти же значения соответственно обозначаются с индексом t+1 в последующие годы. Метод был выбран ввиду его простоты и доступности официальных табличных значений для переменных в правой части выражения.  

Эта методика подходит только для краткосрочного прогнозирования, поскольку не учитывает объем затраченных инвестиций в данный ВЭД и ряд других факторов. 
Кроме того, предполагается, что структура человеческого капитала, необходимая для производства единицы продукции, 
не имеет региональных различий, а определяется только отраслю экономики. 
Данный подход обладает универсальностью, то есть применим для всех субъектов Федерации во всех отраслях экономики. 
\par
Для расчет \textit{ежегодной дополнительной потребности (ЕДП)
\footnote{ЕДП – это ежегодное требуемое приращение к имеющемуся числу занятых до их необходимого количества.} экономики} в кадрах по ВЭД воспользумся формулами ниже. 

\par
$\delta D = \delta L_{e, t+1} + L_{e, t+1}^{-}$
\newline
Слагаемое $\delta L_{e, t+1}$ «на рост» рассчитывается с использованием выражения:
\par
$\delta L_{e, t+1} = L_{e, t+1} - L_{e, t}$
\newline
Слагаемое $L_{e, t+1}^{-}$ показывает численность работников "на замену". Сюда включаются
люди пенсионного возраста. $L_{e, t+1}^{-}$ рассчитывается
с помощью коэффициентов естественного и возрастного выбытия $k_{e,t}$ на основе выражения:
\par
$L_{e, t+1}^{-} =  L_{e, t} * k_{e,t}$
\newline
Коэффициент $k_{e,t}$ оценивается на основе статистических данных о численности работающих
пенсионеров. Подробнее с методом расчета можно ознакомится в статье В. А. Гуртова\cite{estimate-method}.

Что касается расчетов, то для начала требуется определить недостающие коэффициенты в формуле.
Для этого обратимся к данным Федеральной службы государственной статистики. 
Из таблицы "Среднегодовая численность занятых по видам экономической деятельности с 2017
года"\cite{workers-count} мы можем узнать численность занятых в России по видам экономической деятельности
и проценты к итогу. Из таблицы "Валовая добавленная стоимость в основных ценах в соответствии с методологией СНС 2008 (ОКВЭД 2)" берем $X_{i, t}$, а $X_{i, t+1}$ необходимо будет спрогнозировать. В целом, необходимые данные можно без особых проблем найти в открытых источниках.

\clearpage