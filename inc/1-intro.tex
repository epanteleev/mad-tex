\section{Постановка задачи}

\textbf{Задачей} данной работы является прогнозирование потребности экономики в квалифицированных инженерах-мехатроников к 2025 году. 
Работа подразумевает описание выбранного подхода для оценки численности.

\textbf{Результатом} данной работы станет расчитанное число мехатроников в 2025 году. Методику и ссылки на
статистические данные можно будет использовать для прогноза количества кадров в других отраслях экономики. 

\textbf{Итоги} работы могут быть использованы учебными заведениями для коррекции числа учебных мест
специальностей "Мехатроника", "Робототехника", "Автоматизация промышленного предприятия" и т.д.

\clearpage

\section{Результаты}
	В результате данной работы отображены в таблице \ref{res}. \textbf{Для того чтобы покрыть потребности рынка труда, к 2025 году необходимо выпустить около 138 тысячи выпускников.} Такая численность 
	обусловлена высокой оценкой количества работающих людей пенсионного и предпенсионного возраста: 20\% от общего числа трудящихся.
	\par
	По оценкам этой работы, выпускники в основном будут востребованы как замена уходящих на пенсию сотрудников, а не как дополнительные кадры для покрытия новых рабочих мест.
	
	\begin{table}
		\begin{center}
			\caption{Численность работающих в ВЭД, тыс. человек}\label{res}
			\begin{tabular}{ |c|c|c|c| } 
				\hline
				Отрасль                                        & 2020   & 2025   \\
				\hline
				Обрабатывающая пром.                           & 9713,5 & 9459,6 \\
				\hline
				в частности, производство машин и оборудования & 792    & 771    \\
				\hline
			\end{tabular}
		\end{center}
	\end{table}
\clearpage
